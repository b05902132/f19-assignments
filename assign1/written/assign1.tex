
\documentclass[11pt]{article}

\input{../../tex/defs.tex}

% Useful syntax commands
\newcommand{\jarr}[1]{\left[#1\right]}   % \jarr{x: y} = {x: y}
\newcommand{\jobj}[1]{\left\{#1\right\}} % \jobj{1, 2} = [1, 2]
\newcommand{\pgt}[1]{\, > {#1}}          % \pgt{1} = > 1
\newcommand{\plt}[1]{\, < {#1}}          % \plt{2} = < 2
\newcommand{\peq}[1]{\, = {#1}}          % \peq{3} = = 3
\newcommand{\prop}[1]{\langle{#1}\rangle}% \prop{x} = <x>
\newcommand{\matches}[2]{{#1}\sim{#2}}   % \matches{a}{b} = a ~ b
\newcommand{\aeps}{\varepsilon}          % \apes = epsilon
\newcommand{\akey}[2]{.{#1}\,{#2}}       % \akey{s}{a} = .s a
\newcommand{\aidx}[2]{[#1]\,{#2}}        % \aidx{i}{a} = [i] a
\newcommand{\apipe}[1]{\mid {#1}}        % \apipe{a} = | a

% My own command
\newcommand{\eval}[2]{#1 \boldsymbol{\mapsto} #2} % \eval{a}{b} = a |-> b

% Other useful syntax commands:
%
% \msf{x} = x (not italicised)
% \falset = false
% \truet = true
% \tnum = num
% \tbool = bool
% \tstr = str


\begin{document}

\hwtitle
  {Assignment 1}
  {b05902132} %% REPLACE THIS WITH YOUR NAME/ID

\problem{Problem 1}

Part 1:

\begin{alignat*}{1}
\msf{StringProperty}~p_s ::= \qamp \varepsilon \\
\mid \qamp =s \\
\mid \qamp p_s \lor{} p_s
\\
\msf{NumberProperty}~p_n ::= \qamp \varepsilon \\
    \mid \qamp \pgt{n} \\
    \mid \qamp \plt{n} \\
    \mid \qamp \peq{n} \\
    \mid \qamp p_n \lor p_n \\
    \mid \qamp p_n \land p_n \\
\msf{Schema}~\tau ::= \qamp \tstr \prop{p_s} \\
    \mid \qamp \tnum \prop{p_n} \\
    \mid \qamp \tbool \\
    \mid \qamp \jarr{\tau} \\
    \mid \qamp \jobj{(s: \tau)^*} 
\end{alignat*}

Part 2:

% mathpar is the environment for writing inference rules. It takes care of
% the spacing and line breaks automatically. Use "\\" in the premises to
% space out multiple assumptions.
\begin{mathpar}

\ir{S-Bool-False}{\ }{\matches{\falset}{\tbool}}
\ir{S-Bool-True}{\ }{\matches{\truet}{\tbool}}

\ir{S-String-Any}{\ }{\matches{s}{\tstr\prop{\varepsilon}}}
\ir{S-String-Eq}{s'=s}{\matches{s'}{\tstr\prop{=s}}} \\

\ir{S-Num-Any} {\ }{\matches{n}{\tnum\prop{\varepsilon}}}
\ir{S-Num-Eq} {n'=n }{\matches{n'}{\tnum\prop{=n}}}

\ir{S-Num-Lt} {n'<n }{\matches{n'}{\tnum\prop{<n}}}
\ir{S-Num-Gt} {n'>n }{\matches{n'}{\tnum\prop{>n}}} 

\ir{S-Num-And} {\matches{n}{\tnum\prop{p_1}} \\ \matches{n}{\tnum\prop{p_2}} }
{\matches{n}{\tnum\prop{p_1 \land p_2}}} \\
\ir{S-Num-Lor} {\matches{n}{\tnum\prop{p_1}}}{\matches{n}{\tnum\prop{p_1 \lor p_2}}}
\ir{S-Num-Ror} {\matches{n}{\tnum\prop{p_2}}}{\matches{n}{\tnum\prop{p_1 \lor p_2}}}

\ir{S-List}
    {\forall{}i=$0$...|j|-$1$.\matches{j_i}{\tau}}
    {\matches{\jarr{j^*}}{\jarr{\tau}}}
    
\ir{S-Obj}
    {\forall{}s'\in{}s.\matches{j_{s'}}{\tau_{s'}}}
    {
        \matches{\jobj{(s:j)^*}}
                {\jobj{(s:\tau)^*}}
    }
% Inference rules go here

\end{mathpar}

\problem{Problem 2} 

Part 1:

\begin{mathpar}
\ir{A-Indexing}
    {}
    {\eval{([n]a,\jarr{j})}{(a, j_n)}}

\ir{A-Obj}
    {}
    {\eval{(.sa, \jobj{(s, j),...})} {(a, j_s)} }

\ir{A-Pipe}
    {\forall{} i\in $0$..|j| - $1$ . \eval{(a_1a_2, j_i)} (a_2, j'_i)} {\eval {(|a_1a_2, \jarr{j^*})} {(|a_2, \jarr{(j')^*})} }

\ir{A-Pipe-Empty}{}{\eval {(|\varepsilon, \jarr{j})} {(\varepsilon, \jarr{j})} }
% Inference rules go here
\end{mathpar}

Part 2:

\begin{mathpar}
\ir{I-Empty} { } {\matches{\varepsilon}{\tau} }

\ir{I-Obj} {\matches{a}{\tau}} {\matches{sa}{ \jobj{(s:\tau),...} }} 

\ir{I-Indexing}{\matches{a}{\tau}}
   {\matches{[n]a}{\jarr{\tau}}}

\ir{I-Piping} {\matches{a}{\tau}} {\matches{|a}{\jarr{\tau}}}
\end{mathpar}

\textit{Accessor safety}: for all $a, j, \tau$, if $\matches{a}{\tau}$ and $\matches{j}{\tau}$, then there exists a $j'$ such that $\evals{(a, j)}{\aeps, j'}$.

\begin{proof}
% Proof goes here.
\end{proof}

\end{document}
